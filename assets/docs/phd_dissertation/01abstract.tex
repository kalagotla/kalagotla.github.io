%% The text of your abstract and nothing else (other than comments) goes here.
%% It will be single-spaced and the rest of the text that is supposed to go on
%% the abstract page will be generated by the abstractpage environment.  This
%% file should be \input from main.tex.
\cleardoublepage
\phantomsection %to correct hyperlink reference issue
\addcontentsline{toc}{chapter}{Abstract}
\chapter*{Abstract}
% What's the problem?
% Why are we trying to solve it?
% What work has been done towards solving it?

The renewed interest in understanding supersonic and hypersonic regimes made accurate flow measurements essential for advancing aerospace designs and validating high-fidelity computational fluid dynamics (CFD) simulations. Particle Image Velocimetry (PIV) provides a non-intrusive method for visualizing and quantifying complex, high-speed flow fields. However, when applied to supersonic flows, the inherent inertia of tracer particles can lead to significant errors in velocity measurement. These discrepancies become particularly pronounced near shock waves, expansions, vortical structures, and their interactions, where rapid accelerations prevent the particles from faithfully tracking the underlying fluid motion.

This dissertation addresses the critical issue of particle inertia bias in supersonic particle image velocimetry (PIV). First, it introduces the concept of Particle Dynamics History (PDH) to quantify the compounded effects of particle inertia as tracers traverse complex flow topologies. By conducting controlled numerical experiments and post-processing CFD data with a one-way coupled Lagrangian Particle Tracking (LPT) approach, the research reveals the extent of PDH-induced uncertainty across canonical test cases, including oblique shocks and multi-shock systems. These studies demonstrate that local errors due to particle inertia can exceed 5–10\%, resulting in smeared shock interfaces and inaccurate representations of key flow features.

This research develops a comprehensive data-driven modeling framework of three components to mitigate these biases. First, a fine-tunable, object-oriented Lagrangian Particle Tracker (LPT) is created, emphasizing high-fidelity data, advanced drag models, and the ability to transform Lagrangian data into an Eulerian format. Second, syPIV, a synthetic planar PIV image generation project, can help induce the processing-related uncertainty in the velocity data. Third, the Bilateral Inertia Correction Sparse Network (BICSNet), a deep Convolutional Neural Network (CNN), is introduced to correct particle inertia bias in shock-dominated Particle Image Velocimetry (PIV) data. Trained on a wide range of synthetic oblique shock data generated by LPT and syPIV, BICSNet represents a novel approach to tackling this challenge.

The developed tools were initially employed to quantify particle sizing in shock-interaction experiments conducted at Florida State University (FSU). Analysis of this data provided insights into the particle size distribution during PIV experiments, enabling accurate quantification of Stokes number bounds in supersonic PIV studies. Notably, the results revealed a dependency of Stokes number on the window size, underscoring its importance in achieving pixel-level flow field reconstruction, which is an active area of research.

These tools were further applied to Large Eddy Simulation (LES) data of a supersonic jet exhaust case to analyze the impact of Stokes number variations across different flow regimes. The results demonstrate how shock cell structures and shear layer spreading transform in response to changes in particle size, thereby refining the interpretation of experimental particle image velocimetry (PIV) data. These findings lead to the proposal of a new method for reporting the Stokes number, thereby improving the accuracy of non-intrusive yet invasive flow diagnostics, such as PIV, in complex supersonic flows.

Furthermore, PDH-informed synthetic image generation effectively emulates the conditions of supersonic PIV experiments, providing high-fidelity training data for the development of deep learning models. The proposed BICSNet architecture integrates raw PIV images with physics-related parameters, such as Mach and Reynolds numbers, into its learning process. The trained model demonstrates substantial reductions in particle inertia bias, achieving an error reduction of up to 87\% for cases within its training distribution. When applied to the experimental shock interaction data, BICSNet significantly improved velocity estimation up to 58\% in the tested cases, particularly at the shock fronts, further validating its potential for experimental workflows. Although model performance diminishes in out-of-distribution scenarios, these results establish a promising pathway toward developing universal corrections applicable to a broad range of flow conditions. 

The findings underscore the transformative potential of integrating physics-based modeling, synthetic data generation, and machine learning to refine PIV diagnostics in supersonic and hypersonic environments. By mitigating particle inertia bias, this research will yield more reliable experimental data, an improved understanding of complex high-speed flow physics, and more robust validation of advanced aerospace designs and computational models.


\pagebreak
\hspace{0pt}
\vfill
\begin{center}
Copyright \copyright{} \yourName{} \yourYear{}
\end{center}
\vfill
\hspace{0pt}
\pagebreak