\chapter{Conclusions and Future Work}\label{ch:conclusions}

\section*{Conclusion}

Particle Inertia Bias is a longstanding source of systematic uncertainty in particle-based velocity measurements such as Particle Image Velocimetry (PIV) and Laser Doppler Velocimetry (LDV). Accurate quantification and mitigation of this bias are critical for interpreting data from these experiments, especially in supersonic or shock-laden flows. This dissertation presented a novel framework that leverages machine learning to address particle inertia bias in PIV experiments focused on oblique shocks.

A key component of this framework is a one-way coupled Lagrangian Particle Tracking (LPT) code developed from scratch in the Python programming language. This code has been tested and validated to accurately track the behavior of tracer particles in supersonic PIV experiments. Since particle tracking is time-consuming and computationally expensive, a new methodology has been introduced to transfer Lagrangian data into an Eulerian format. This methodology eliminates the need to track the same number of particles typically tracked in PIV, producing a high-fidelity Eulerian field that complements the particle-field data obtained from PIV experiments. The transformed dataset, called Particle Dynamics History (PDH), preserves the essential particle trajectory information and can generate synthetic PIV images using the synthetic PIV (\texttt{syPIV}) code. This process produces two corresponding synthetic image sets: one free from particle inertia bias and another incorporating particle inertia bias informed by PDH. These paired datasets then serve as inputs (with bias) and labels (without bias) for subsequent machine learning workflows.

The developed codes were used to study particle behavior in complex shock-laden flows. A shock-interaction case study was conducted to understand the impact of PDH. For this purpose, a Mach 3 flow with converging shocks was designed. Using the NASA OVERFLOW code, a Reynolds-Averaged Navier-Stokes (RANS) simulation was performed to obtain the mean velocity field. This particle tracking study demonstrated that as the shocks converged, the particles introduced did not have sufficient length to relax to the post-shock velocity of the first shock and instead passed through a second successive shock, resulting in increased inertia bias in this region. For a case where the shocks are close, the bias due to PDH was found to be 6.41\% in comparison to 5.5\% if the particle had relaxed. Furthermore, generating synthetic images from the PDH-informed data highlighted the effect of analysis bias, which became more significant as the shocks converged, making it appear in the data as if it were a single shock.

Furthermore, the LPT code was utilized to develop a novel methodology for quantifying particle size distribution following a PIV run. This helps us understand the actual particle distribution during a PIV experiment. The data obtained showed that the particle distribution is skewed compared to their response times, highlighting the non-linear nature of the drag acting on these particles. This methodology was later tested on a shock-interaction PIV study conducted at Florida State University (FSU). The results revealed that the particle distribution was skewed, and significant clusters of particle groups were identified, quantifying the effect of preferential concentration. The Stokes number study showed that the mean is well below 0.05, allowing for the study of large-scale interactions. This highlights one of the use cases for the current open-source code. The identified particle sizes were used to showcase that the PIV data agree better with the corresponding CFD data (a RANS simulation conducted using adaptive meshing in Ansys Fluent) when the particle inertia bias effect is applied. An improvement of up to 85\% in velocity comparison is showcased in the shock interaction region between CFD and PIV data, indicating the significance of considering particle inertia bias.

Beyond oblique shocks, the code was applied to a complex jet exhaust scenario that employed a Large Eddy Simulation (LES) for the mean flow field. Several particle sizes were introduced at various domain boundaries to create PDH-informed synthetic images aligned with the experimental parameters. Analysis revealed significant particle inertia bias near the jet centerline, highlighting that the LES-predicted shock strength is more accurate than previously inferred. Additionally, it was found that a nominally $20\mu m$ particle diameter best matched experimental PIV observations in the near jet exhaust region, suggesting a strong particle agglomeration effect --- an issue commonly encountered in shock-laden flows.

Finally, this dissertation introduced the Bilateral Inertia-Correction Sparse Network (BICSNet), an encoder-decoder style convolutional neural network designed to transform particle inertia-biased PIV images into images without inertia bias. BICSNet was trained on 600 simulated oblique shock flow cases, each containing images generated by varying shock locations, particle sizes, densities, and concentrations. A significant reduction in inertia bias was observed for in-distribution test cases. However, the network did not accurately represent the physical features in the out-of-distribution cases. This model was also applied to experimental PIV data, significantly improving shock front detection. The model, however, did not perform as effectively in the shock interaction region, where the shocks are closely packed together. The results overall demonstrate the promise of deep learning in systematically reducing inertia bias, thereby improving the accuracy of PIV-based flow measurements. This work lays the foundation for developing bias-aware machine learning models, which could transform experimental workflows with minimal computational effort.

\section*{Future Work}

While this research marks a significant advancement in addressing particle inertia bias, several pathways remain for further exploration:

\begin{itemize}
    \item \textbf{Study of turbulence}: The LPT code could be used to study and quantify the effects of particles on turbulence quantification in PIV, which is a significant area of interest. This requires expanding the current modules to accommodate the unsteady nature of the flow, which has been tested and implemented in the existing code. Refer to Appendix \ref{ch:appendices} for more details on the unsteady implementation.
    \item \textbf{Extension to stereo/tomo-PIV:} The current work explored PDH effects in planar PIV. Adapting and validating the LPT-PDH workflow and BICSNet framework for stereo/tomo-PIV, where particle inertia bias may have even more complex manifestations.
    \item \textbf{Non-spherical particles:} Particle agglomeration is reported in strong turbulent zones such as oblique shocks in PIV. Agglomerates are known to be non-spherical. Adapting the LPT framework to tackle this would provide more insights into the effect of inertia bias.
    \item \textbf{Expanded Training Sets:} Including more varied oblique shock conditions in the training dataset could improve model robustness and reduce out-of-distribution errors.
    \item \textbf{Advanced Network Architectures:} Introducing skip connections or attention mechanisms in BICSNet will enable more effective feature extraction and allow us to study higher-order flow statistics.
    \item \textbf{Physics-Informed Training:} Incorporating additional physics-based constraints, such as the processed velocity fields, into the training objective can enhance interpretability and stability in extreme flow regimes.
    \item \textbf{Application to Real Experiments:} Validating BICSNet across a broader range of experimental datasets that include shocks will be essential to showcase its effectiveness in practical settings fully.
\end{itemize}

In conclusion, the tools and methodologies developed in this dissertation provide a rigorous framework for understanding and mitigating particle inertia bias in supersonic particle image velocimetry (PIV) experiments. From shock-interaction analyses to jet exhaust studies, the combined use of LPT, PDH-informed synthetic images, and machine learning offers a powerful path toward more accurate experimental measurements. As supersonic and hypersonic applications continue to expand, these approaches have the potential to significantly advance both fundamental research and technological innovation in high-speed aerodynamics.
